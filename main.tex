\documentclass[a4paper, 12pt]{report}
\usepackage{amsmath, amssymb, amstext, amsthm, amssymb, leftidx, siunitx}
\usepackage[
    type={CC},
    modifier={by-nc-sa},
    version={4.0},
    lang={it},
]{doclicense}
\usepackage[italian]{babel}
\usepackage{tikz}
\usepackage{pgfplots}
\pgfplotsset{compat=1.15}

\theoremstyle{definition}
\newtheorem{definition}{Definizione}[chapter]

\theoremstyle{remark}
\newtheorem{example}{Esempio}[definition]
\newtheorem{remark}{Osservazione}[definition]

\theoremstyle{plain}
\newtheorem{theorem}{Teorema}[chapter]
\newtheorem{corollary}{Corollario}[theorem]

\newcommand{\N}{\mathbb{N}}
\newcommand{\Z}{\mathbb{Z}}
\newcommand{\R}{\mathbb{R}}
\newcommand{\C}{\mathbb{C}}

\begin{document}

\begin{titlepage}
    \begin{center}
        \vspace*{5cm}
        \Huge{Formulario di Fisica}\\[1cm]
        \Large{Mario Merlo}\\
        \Large{https://www.github.com/MrVideo}\\
        \Large{Politecnico di Milano}\\[7,5cm]
    \end{center}
    \doclicenseThis
\end{titlepage}

\chapter{Strumenti di base}

\section{Grandezze fisiche ed indici di stato fisico}

\begin{definition}
    Una {\bf Grandezza Fisica} è un insieme di enti del mondo, non meglio precisati, tra i quali possiamo stabilire:
    \begin{itemize}
        \item Una relazione d'ordine
        \item Operazioni di somma o differenza
        \item Operazioni di prodotto per scalari o per qualsiasi numero reale
        \item Operazioni di prodotto scalare tra gli enti
    \end{itemize}
\end{definition}

\begin{example}
    Alcune grandezze fisiche:
    \begin{itemize}
        \item Lunghezza: $[L]$
        \item Massa: $[M]$
        \item Intervallo di tempo: $[T]$
        \item Forza: $[F]$
        \item Lavoro: $[W]$
    \end{itemize}
\end{example}

\begin{definition}
    Un {\bf Indice di Stato Fisico} è un insieme di enti del mondo non meglio precisato tra i quali possiamo stabilire:
    \begin{itemize}
        \item Una relazione di uguaglianza
        \item Operazioni di somma e differenza
        \item Operazioni di prodotto per scalari
        \item Operazioni di prodotto scalare
    \end{itemize}
\end{definition}

\begin{example}
    Un esempio di indice di stato fisico sono le coordinate geometriche:
    \begin{center}
        \begin{tikzpicture}
            \draw (0,0) -- (5,0);
            \draw (1,0) circle [radius=1pt];
            \draw (1,0) node[anchor=south] {$P_1$};
            \draw (4,0) circle [radius=1pt];
            \draw (4,0) node[anchor=south] {$P_2$};
        \end{tikzpicture}
    \end{center}
    Noi possiamo individuare due punti su una retta, ma non possiamo ordinarli...
    \begin{center}
        \begin{tikzpicture}
            \draw[->] (0,0) -- (5,0);
            \draw (1,0) circle [radius=1pt];
            \draw (1,0) node[anchor=south] {$P_1$};
            \draw (4,0) circle [radius=1pt];
            \draw (4,0) node[anchor=south] {$P_2$};
        \end{tikzpicture}
    \end{center}
    ... Ma aggiungendo un verso, una convenzione, possiamo farlo.
\end{example}

\begin{remark}
    Grandezze fisiche ed indici di stato fisico presentano un {\bf rapporto di corrispondenza}:
    \begin{center}
        \begin{tikzpicture}
            \draw (0,0) -- (5,0);
            \draw (1,0) circle [radius=1pt];
            \draw (1,0) node[anchor=south] {$P_1$};
            \draw (4,0) circle [radius=1pt];
            \draw (4,0) node[anchor=south] {$P_2$};
            \draw (1,0) node[anchor=north] {$x_1$};
            \draw (4,0) node[anchor=north] {$x_2$};
            \draw[red] (1,0) -- (4,0);
        \end{tikzpicture}
    \end{center}
    Possiamo definire una {\bf distanza} tra $P_1$ e $P_2$:
    \begin{center}
        $L = |x_2 - x_1|$\\
    \end{center}
    Questa distanza è una {\bf grandezza fisica}, nello specifico una lunghezza.
\end{remark}

\section{Misure di una grandezza}

La misura di una grandezza è una {\it procedura convenuta e codificata}, tipicamente strumentale, finalizzata a caratterizzare quantitativamente una grandezza fisica attraverso un numero reale.
Viene scelto, a questo scopo, un elemento all'interno della classe di grandezza, detto {\bf unità di misura}, $U$.

\begin{definition}
    La misura di una grandezza $G$ è il numero di volte in cui $U$ è contenuta in $G$: $G = nU, n \in \R$
\end{definition}

\begin{remark}
    Come cambiare unità di misura?
    $U \rightarrow U', n = \frac{G}{U}, n' = \frac{G}{U'}$
    Devo confrontare $n$ ed $n'$. Come?
    $n' = \frac{G}{U'} = \frac{G}{U} \cdot \frac{U}{U'} = n \cdot \frac{U}{U'}$, dove $\frac{U}{U'}$ è detto {\bf fattore di ragguaglio}.
\end{remark}

\subsection{Grandezze fisiche fondamentali}

\begin{definition}
    Si definiscono {\it fondamentali} le grandezze fisiche per cui il procedimento di misura è {\bf diretto}, ossia per cui non servono altre grandezze.
\end{definition}

Nel {\bf Sistema Internazionale} (SI), per la meccanica, esse sono:
\begin{enumerate}
    \item {\bf Intervallo di tempo} $[T]$:
        \begin{itemize}
            \item {\bf Unità di misura}: secondo (\SI{}{s})
            \item $\SI{1}{s} = \SI{9192631770}{}$ cicli dell'oscillazione elettromagnetica emessa nella transizione tra due livello iperfini dello stato fondamentale dell'isotopo più stabile del Cesio ${}^{133}$Cs.
        \end{itemize}
    \item {\bf Lunghezza} $[L]$:
        \begin{itemize}
            \item {\bf Unità di misura}: metro (\SI{}{m})
            \item $\SI{1}{m} = $ lunghezza del tratto percorso dala luce nel vuoto nell'intervallo di tempo pari a $\frac{1}{\SI{299792458}{}}\SI{}{s}$
        \end{itemize}
    \item {\bf Massa} $[M]$:
    \begin{itemize}
        \item {\bf Unità di misura}: chilogrammo (\SI{}{kg})
        \item $\SI{1}{kg} = $ massa del campione di platino-iridio conservato a Sévres.
    \end{itemize}
\end{enumerate}

Queste grandezze sono riprese dal vecchio sistema $mks$.

\subsection{Grandezze fisiche derivate}

\begin{definition}
    Si definiscono {\it derivate} le grandezze fisiche ottenute con un processo "indiretto" che le fa derivare da grandezze fondamentali.
\end{definition}

\begin{example}
    Alcuni esempi:
    \begin{itemize}
        \item {\bf Velocità} $[v]$: $v \triangleq \frac{l}{\Delta t}$
        \item {\bf Accelerazione} $[a]$: $a \triangleq \frac{\Delta v}{\Delta t}$
    \end{itemize}
\end{example}

\subsection{Grandezze supplementari}

Esistono due importanti grandezze supplementari:
\begin{itemize}
    \item {\bf Angolo piano}: unità di misura: {\bf Radiante} (\SI{}{rad})
    \item {\bf Angolo solido}: unità di misura: {\bf Steradiante} (\SI{}{sr})
\end{itemize}

\begin{center}
    \begin{tikzpicture}
        \draw (0,0) circle[radius=1];
        \draw[->] (0,-1.5) -- (0,1.5);
        \draw (0,2) node[anchor=east] {$y$};
        \draw[->] (-1.5,0) -- (1.5,0);
        \draw (2,0) node[anchor=north] {$x$};
        \draw[blue,thick] (0,-1) -- (0,0);
        \draw (0,-0.5) node[anchor=east] {$R$};
        \filldraw[red,thick,fill=red!20!white] (0,0) -- (1,0) arc [start angle=0, end angle=57.29, radius=1] -- cycle;
        \draw (0,0) node[anchor=south east] {$\alpha$};
    \end{tikzpicture}
\end{center}

$\alpha$ è l'angolo sotteso ad un arco di lunghezza pari al raggio $R$ della circonferenza. $\alpha = \SI{1}{rad}$.

% Placeholder for steradian graph
\begin{center}
    \begin{tikzpicture}
        \begin{axis}
            [
            width=6cm,height=6cm,
            axis equal,enlargelimits=false,
            axis lines=none,
            domain=0:180,samples=21,
            y domain=0:360,samples y=21,
            colormap/blackwhite,
            view={100}{10},
            ]
            \addplot3
            [
                surf,
                z buffer=sort,
                shader=flat,
                point meta={acos(z/sqrt(x*x+y*y+z*z)) + atan2(y,x)}
            ]   (
                {sin(x)*cos(y)},
                {sin(x)*sin(y)},
                {cos(x)}
            );
        \end{axis}
    \end{tikzpicture}
\end{center}

\end{document}