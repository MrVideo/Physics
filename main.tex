\documentclass[a4paper, 12pt]{report}
\usepackage{amsmath, amssymb, amstext, amsthm, amssymb}
\usepackage[
    type={CC},
    modifier={by-nc-sa},
    version={4.0},
    lang={it},
]{doclicense}
\usepackage[italian]{babel}
\usepackage{tikz}

\theoremstyle{definition}
\newtheorem{definition}{Definizione}[chapter]

\theoremstyle{remark}
\newtheorem{example}{Esempio}[definition]
\newtheorem{remark}{Osservazione}[definition]

\theoremstyle{plain}
\newtheorem{theorem}{Teorema}[chapter]
\newtheorem{corollary}{Corollario}[theorem]

\newcommand{\N}{\mathbb{N}}
\newcommand{\Z}{\mathbb{Z}}
\newcommand{\R}{\mathbb{R}}
\newcommand{\C}{\mathbb{C}}

\begin{document}


\begin{titlepage}
    \begin{center}
        \vspace*{5cm}
        \Huge{Formulario di Fisica}\\[1cm]
        \Large{Mario Merlo}\\
        \Large{https://www.github.com/MrVideo}\\
        \Large{Politecnico di Milano}\\[7,5cm]
    \end{center}
    \doclicenseThis
\end{titlepage}

\chapter{Strumenti di base}

\begin{definition}
    Una {\bf Grandezza Fisica} è un insieme di enti del mondo, non meglio precisati, tra i quali possiamo stabilire:
    \begin{itemize}
        \item Una relazione d'ordine
        \item Operazioni di somma o differenza
        \item Operazioni di prodotto per scalari o per qualsiasi numero reale
        \item Operazioni di prodotto scalare tra gli enti
    \end{itemize}
\end{definition}

\begin{example}
    Alcune grandezze fisiche:
    \begin{itemize}
        \item Lunghezza: $[L]$
        \item Massa: $[M]$
        \item Intervallo di tempo: $[T]$
        \item Forza: $[F]$
        \item Lavoro: $[W]$
    \end{itemize}
\end{example}

\begin{definition}
    Un {\bf Indice di Stato Fisico} è un insieme di enti del mondo non meglio precisato tra i quali possiamo stabilire:
    \begin{itemize}
        \item Una relazione di uguaglianza
        \item Operazioni di somma e differenza
        \item Operazioni di prodotto per scalari
        \item Operazioni di prodotto scalare
    \end{itemize}
\end{definition}

\begin{example}
    Un esempio di indice di stato fisico sono le coordinate geometriche:\\\\
    \begin{tikzpicture}
        \draw (0,0) -- (5,0);
        \draw (1,0) circle [radius=1pt];
        \draw (1,0) node[anchor=south] {$P_1$};
        \draw (4,0) circle [radius=1pt];
        \draw (4,0) node[anchor=south] {$P_2$};
    \end{tikzpicture}
    Noi possiamo individuare due punti su una retta, ma non possiamo ordinarli...\\
    \begin{tikzpicture}
        \draw[->] (0,0) -- (5,0);
        \draw (1,0) circle [radius=1pt];
        \draw (1,0) node[anchor=south] {$P_1$};
        \draw (4,0) circle [radius=1pt];
        \draw (4,0) node[anchor=south] {$P_2$};
    \end{tikzpicture}
    ... Ma aggiungendo un verso, una convenzione, possiamo farlo.
\end{example}

\end{document}