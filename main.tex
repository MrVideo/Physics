\documentclass[a4paper, 12pt]{report}
\usepackage{amsmath, amssymb, amstext, amsthm, amssymb, leftidx, siunitx, systeme, cancel, hyperref}
\usepackage[
    type={CC},
    modifier={by-nc-sa},
    version={4.0},
    lang={it},
]{doclicense}
\usepackage[italian]{babel}
\usepackage{tikz}
\usepackage{pgfplots}
\pgfplotsset{compat=1.15}

\hypersetup{
    colorlinks,
    citecolor=black,
    filecolor=black,
    linkcolor=black,
    urlcolor=black
}

\theoremstyle{definition}
\newtheorem{definition}{Definizione}[section]

\theoremstyle{remark}
\newtheorem*{example}{Esempio}
\newtheorem*{remark}{Osservazione}
\newtheorem*{prop}{Proprietà}

\theoremstyle{plain}
\newtheorem{theorem}{Teorema}[section]
\newtheorem{corollary}{Corollario}[theorem]

\renewcommand{\theequation}{\arabic{chapter}.\arabic{section}.\arabic{equation}}

\newcommand{\N}{\mathbb{N}}
\newcommand{\Z}{\mathbb{Z}}
\newcommand{\R}{\mathbb{R}}
\newcommand{\C}{\mathbb{C}}

\begin{document}

\begin{titlepage}
    \begin{center}
        \vspace*{5cm}
        \Huge{Formulario di Fisica}\\[1cm]
        \Large{Mario Merlo}\\
        \Large{https://www.github.com/MrVideo}\\
        \Large{Politecnico di Milano}\\[7,5cm]
    \end{center}
    \doclicenseThis
\end{titlepage}

\tableofcontents

\chapter{Strumenti di base}

\section{Grandezze fisiche ed indici di stato fisico}

\begin{definition}
    Una {\bf Grandezza Fisica} è un insieme di enti del mondo, non meglio precisati, tra i quali possiamo stabilire:
    \begin{itemize}
        \item Una relazione d'ordine
        \item Operazioni di somma o differenza
        \item Operazioni di prodotto per scalari o per qualsiasi numero reale
        \item Operazioni di prodotto scalare tra gli enti
    \end{itemize}
\end{definition}

\begin{example}
    Alcune grandezze fisiche:
    \begin{itemize}
        \item Lunghezza: $[L]$
        \item Massa: $[M]$
        \item Intervallo di tempo: $[T]$
        \item Forza: $[F]$
        \item Lavoro: $[W]$
    \end{itemize}
\end{example}

\begin{definition}
    Un {\bf Indice di Stato Fisico} è un insieme di enti del mondo non meglio precisato tra i quali possiamo stabilire:
    \begin{itemize}
        \item Una relazione di uguaglianza
        \item Operazioni di somma e differenza
        \item Operazioni di prodotto per scalari
        \item Operazioni di prodotto scalare
    \end{itemize}
\end{definition}

\begin{example}
    Un esempio di indice di stato fisico sono le coordinate geometriche:
    \begin{center}
        \begin{tikzpicture}
            \draw (0,0) -- (5,0);
            \draw (1,0) circle [radius=1pt];
            \draw (1,0) node[anchor=south] {$P_1$};
            \draw (4,0) circle [radius=1pt];
            \draw (4,0) node[anchor=south] {$P_2$};
        \end{tikzpicture}
    \end{center}
    Noi possiamo individuare due punti su una retta, ma non possiamo ordinarli...
    \begin{center}
        \begin{tikzpicture}
            \draw[->] (0,0) -- (5,0);
            \draw (1,0) circle [radius=1pt];
            \draw (1,0) node[anchor=south] {$P_1$};
            \draw (4,0) circle [radius=1pt];
            \draw (4,0) node[anchor=south] {$P_2$};
        \end{tikzpicture}
    \end{center}
    ... Ma aggiungendo un verso, una convenzione, possiamo farlo.
\end{example}

\begin{remark}
    Grandezze fisiche ed indici di stato fisico presentano un {\bf rapporto di corrispondenza}:
    \begin{center}
        \begin{tikzpicture}
            \draw (0,0) -- (5,0);
            \draw (1,0) circle [radius=1pt];
            \draw (1,0) node[anchor=south] {$P_1$};
            \draw (4,0) circle [radius=1pt];
            \draw (4,0) node[anchor=south] {$P_2$};
            \draw (1,0) node[anchor=north] {$x_1$};
            \draw (4,0) node[anchor=north] {$x_2$};
            \draw[red] (1,0) -- (4,0);
        \end{tikzpicture}
    \end{center}
    Possiamo definire una {\bf distanza} tra $P_1$ e $P_2$:
    \begin{center}
        $L = |x_2 - x_1|$\\
    \end{center}
    Questa distanza è una {\bf grandezza fisica}, nello specifico una lunghezza.
\end{remark}

\section{Misure di una grandezza}

La misura di una grandezza è una {\it procedura convenuta e codificata}, tipicamente strumentale, finalizzata a caratterizzare quantitativamente una grandezza fisica attraverso un numero reale.
Viene scelto, a questo scopo, un elemento all'interno della classe di grandezza, detto {\bf unità di misura}, $U$.

\begin{definition}
    La misura di una grandezza $G$ è il numero di volte in cui $U$ è contenuta in $G$: $G = nU, n \in \R$
\end{definition}

\begin{remark}
    Come cambiare unità di misura?
    $U \rightarrow U', n = \frac{G}{U}, n' = \frac{G}{U'}$
    Devo confrontare $n$ ed $n'$. Come?
    $n' = \frac{G}{U'} = \frac{G}{U} \cdot \frac{U}{U'} = n \cdot \frac{U}{U'}$, dove $\frac{U}{U'}$ è detto {\bf fattore di ragguaglio}.
\end{remark}

\subsection{Grandezze fisiche fondamentali}

\begin{definition}
    Si definiscono {\it fondamentali} le grandezze fisiche per cui il procedimento di misura è {\bf diretto}, ossia per cui non servono altre grandezze.
\end{definition}

Nel {\bf Sistema Internazionale} (SI), per la meccanica, esse sono:
\begin{enumerate}
    \item {\bf Intervallo di tempo} $[T]$:
        \begin{itemize}
            \item {\bf Unità di misura}: secondo (\SI{}{s})
            \item $\SI{1}{s} = \SI{9192631770}{}$ cicli dell'oscillazione elettromagnetica emessa nella transizione tra due livello iperfini dello stato fondamentale dell'isotopo più stabile del Cesio ${}^{133}$Cs.
        \end{itemize}
    \item {\bf Lunghezza} $[L]$:
        \begin{itemize}
            \item {\bf Unità di misura}: metro (\SI{}{m})
            \item $\SI{1}{m} = $ lunghezza del tratto percorso dala luce nel vuoto nell'intervallo di tempo pari a $\frac{1}{\SI{299792458}{}}\SI{}{s}$
        \end{itemize}
    \item {\bf Massa} $[M]$:
    \begin{itemize}
        \item {\bf Unità di misura}: chilogrammo (\SI{}{kg})
        \item $\SI{1}{kg} = $ massa del campione di platino-iridio conservato a Sévres.
    \end{itemize}
\end{enumerate}

Queste grandezze sono riprese dal vecchio sistema $mks$.

\subsection{Grandezze fisiche derivate}

\begin{definition}
    Si definiscono {\it derivate} le grandezze fisiche ottenute con un processo "indiretto" che le fa derivare da grandezze fondamentali.
\end{definition}

\begin{example}
    Alcuni esempi:
    \begin{itemize}
        \item {\bf Velocità} $[v]$: $v \triangleq \frac{l}{\Delta t}$
        \item {\bf Accelerazione} $[a]$: $a \triangleq \frac{\Delta v}{\Delta t}$
    \end{itemize}
\end{example}

\subsection{Grandezze supplementari}

Esistono due importanti grandezze supplementari:
\begin{itemize}
    \item {\bf Angolo piano}: unità di misura: {\bf Radiante} (\SI{}{rad})
    \item {\bf Angolo solido}: unità di misura: {\bf Steradiante} (\SI{}{sr})
\end{itemize}

\begin{center}
    \begin{tikzpicture}
        \draw (0,0) circle[radius=1];
        \draw[->] (0,-1.5) -- (0,1.5);
        \draw (0,2) node[anchor=east] {$y$};
        \draw[->] (-1.5,0) -- (1.5,0);
        \draw (2,0) node[anchor=north] {$x$};
        \draw[blue,thick] (0,-1) -- (0,0);
        \draw (0,-0.5) node[anchor=east] {$R$};
        \filldraw[red,thick,fill=red!20!white] (0,0) -- (1,0) arc [start angle=0, end angle=57.29, radius=1] -- cycle;
        \draw (0,0) node[anchor=south east] {$\alpha$};
    \end{tikzpicture}
\end{center}

$\alpha$ è l'angolo sotteso ad un arco di lunghezza pari al raggio $R$ della circonferenza. $\alpha = \SI{1}{rad}$.

% Placeholder for steradian graph
\begin{center}
    \begin{tikzpicture}
        \begin{axis}
            [
            width=6cm,height=6cm,
            axis equal,enlargelimits=false,
            axis lines=none,
            domain=0:180,samples=21,
            y domain=0:360,samples y=21,
            colormap/blackwhite,
            view={100}{10},
            ]
            \addplot3
            [
                surf,
                z buffer=sort,
                shader=flat,
                point meta={acos(z/sqrt(x*x+y*y+z*z)) + atan2(y,x)}
            ]   (
                {sin(x)*cos(y)},
                {sin(x)*sin(y)},
                {cos(x)}
            );
        \end{axis}
    \end{tikzpicture}
\end{center}

L'angolo sotteso ad una calotta sferica di area $A = R^2$ è pari ad \SI{1}{sr}.

\subsection{Grandezze scalari e grandezze vettoriali}

Esistono fondamentalmente due tipi di grandezze fisiche:
\begin{itemize}
    \item {\bf Grandezze scalari}: grandezze la cui misura è definita da un numero reale
    \item {\bf Grandezze vettoriali}: grandezze la cui misura è definita da più di un numero reale (generalmente 2 o 3).
\end{itemize}

\begin{example}
    Alcune grandezze scalari e vettoriali:
    \begin{itemize}
        \item {\bf Scalari}: lunghezza, massa, intervallo di tempo.
        \item {\bf Vettoriali}: velocità, accelerazione, forza.
    \end{itemize}
\end{example}

Per definire le grandezze vettoriali si usano o tre numeri reali o una terna composta da:
\begin{enumerate}
    \item {\bf Modulo}: numero reale non negativo ($x \in \R, x \geq 0$)
    \item {\bf Direzione}: una direzione dello spazio
    \item {\bf Verso}: un verso della direzione suddetta
\end{enumerate}

\begin{definition}
    La {\bf dimensione} di una grandezza fisica è l'esponente con cui $A$ compare nella definizione di $G$.
\end{definition}

\begin{example}
    $v \triangleq \frac{l}{\Delta t}$
    $dim = 1$ rispetto a $[L]$ e $dim = -1$ rispetto a $[T]$.
    Le parentesi quadre sono dette {\bf simbolo dimensionale}.
\end{example}

\begin{definition}
    Si definisce {\bf legge fisica} una relazione {\it quantitativa} tra grandezze fisiche espressa attraverso un'equazione tra le misure delle grandezze considerate. 
\end{definition}

\begin{definition}
    {\bf Principio di omogeneità}: le equazioni che rappresentano le leggi fisiche devono essere scritte in modo da risultare indipendenti dalle unità di misura scelte per le diverse grandezze fisiche.
\end{definition}

\begin{remark}
    Condizione necessaria e sufficiente affinché l'equazione che esprime una legge fisica rispetti il principio di omogeneità è che {\bf i membri dell'equazione abbiano le stesse dimensioni} in termini di grandezze fondamentali.
\end{remark}

\chapter{Cinematica}

La cinematica è la branca della fisica che studia il moto {\it in senso descrittivo}, prescindendo dalle cause del moto.

\section{Cinematica scalare}

\begin{definition}
    {\bf Punto materiale}: è un corpo che possiede:
    \begin{enumerate}
        \item dimensioni piccole rispetto alla scala delle lunghezze in esame
        \item una struttura interna {\it non coinvolta} o {\it trascurabile} nel fenomeno in esame
    \end{enumerate}
\end{definition}

\begin{definition}
    {\bf Legge oraria del moto}: equazione che definisce istante per istante, a partire da un dato istante iniziale $t_0$, le coordinate spaziali del punto materiale rispetto ad un dato sistema di riferimento.
\end{definition}

\begin{example}
    Una generica legge oraria potrebbe essere:
    \begin{center}
        $\systeme*{x = x(t), y = y(t), z = z(t)}, \forall t \geq t_0, x, y, z$ coordinate cartesiane del punto.
    \end{center}
\end{example}

\begin{definition}
    {\bf Traiettoria}: curva dello spazio costituita da tutti e soli i punti geometrici occupato dal punto materiale durante il suo moto.
\end{definition}

\begin{remark}
    Il moto può anche essere classificato in base alla sua traiettoria:
    \begin{itemize}
        \item {\it Rettilineo}: la traiettoria giace su una {\bf retta}.
        \item {\it Circolare}: la traiettoria giace su una {\bf circonferenza}.
        \item {\it Parabolico}: la traiettoria giace su una {\bf parabola}.
        \item {\it Curvilineo}: la traiettoria giace su una {\bf curva generica} (caso generale).
    \end{itemize}
\end{remark}

\begin{definition}
    {\bf Ascissa curvilinea}: coordinata {\it unidimensionale} curvilinea del punto materiale definita sulla sua traiettoria.
\end{definition}

\begin{remark}
    Si può definire la {\bf legge oraria dell'ascissa curvilinea} con l'equazione $s = s(t)$, che definisce il valore di $s(t)$ istante per istante.
\end{remark}

\begin{definition}
    {\bf Velocità scalare media tra due istanti}
    \begin{equation}
        v_m(t_1, t_2) \triangleq \frac{s(t_2) - s(t_1)}{t_2 - t_1} \equiv \frac{\Delta s(t_1, t_2)}{\Delta t}
    \end{equation}
\end{definition}

\begin{remark}
    Se risulta $v_m(t_1, t_2) = \overline{v}$ costante, cioè indipendente dalla scelta degli istanti $t_1$ e $t_2$, allora si dice che il moto è {\bf uniforme}.
\end{remark}

\begin{definition}
    {\bf Legge dell'ascissa curvilinea per il moto uniforme}
    \begin{equation}
        s(t) = s(0) + \overline{v}t
    \end{equation}
\end{definition}

Se, come accade più in generale, $v_m(t_1, t_2)$ {\it non è costante}, come possiamo caratterizzare la velocità del moto {\it in un dato istante}?

\begin{definition}
    {\bf Velocità scalare istantanea}
    \begin{equation}
        v(t) \triangleq \lim_{\Delta t \to 0} v_m(t, t + \Delta t)  
    \end{equation}
\end{definition}

\begin{remark}
    \begin{equation}
        \lim_{\Delta t \to 0} v_m(t, t + \Delta t) = \lim_{\Delta t \to 0} \frac{s(t + \Delta t) - s(t)}{\cancelto{0}{t} + \Delta t - \cancelto{0}{t}} = \lim_{\Delta t \to 0} \frac{s(t + \Delta t) - s(t)}{\Delta t} = \frac{ds(t)}{dt}
    \end{equation}
    Quindi, per definizione, $v(t) = \frac{ds(t)}{dt}$.
\end{remark}

\begin{prop}
    Sia nota la legge $v = v(t), \forall t \in [t_1, t_2]$ e sia nota $s$ nell'istante $t_0 \in [t_1, t_2]$, allora {\bf è determinabile la legge dell'ascissa curvilinea}: $s = s(t), \forall t \in [t_1, t_2]$.
\end{prop}

\begin{proof}
    Poiché $v(t) = \frac{ds(t)}{dt}$, allora $ds(t) = v(t)dt$. Integrando entrambi i membri tra due estremi opportuni $t_0$, $t \in [t_1, t_2]$:
    \begin{gather}
        \nonumber \int_{t_0}^{t} ds(t') = \int_{t_0}^{t} v(t')dt'\\
        \nonumber s(t) - s(t_0) = \int_{t_0}^{t} v(t')dt'\\
        s(t) = s(t_0) + \int_{t_0}^{t} v(t')dt'
    \end{gather}
    $s(t_0)$ è noto e l'integrale è determinato
\end{proof}

\begin{definition}
    {\bf Accelerazione scalare media}
    \begin{equation}
        a_m(t_1, t_2) \triangleq \frac{v(t_2) - v(t_1)}{t_2 - t_1} = \frac{\Delta v(t_1, t_2)}{\Delta t}
    \end{equation}
\end{definition}

\begin{definition}
    {\bf Moto uniformemente accelerato}: moto in cui $a_m$ tra due istanti qualsiasi è costante:
    \begin{equation}
        \forall t_1, t_2, a_m(t_1, t_2) = \overline{a} \mbox{ costante}
    \end{equation}
\end{definition}

\begin{remark}
    \begin{gather}
        \nonumber a_m(t_1, t_2) \triangleq \frac{v(t_2) - v(t_1)}{t_2 - t_1}, t_1 = 0, t_2 = t \mbox{ generico}\\
    \end{gather}
\end{remark}

\end{document}